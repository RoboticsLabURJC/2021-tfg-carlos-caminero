\chapter{Objetivos y Metodología de Trabajo}
\label{cap:capitulo2}

La importancia de los robots de servicio ha ido aumentando en los últimos años, sobre todo en campos como la salud y el entretenimiento. Con la era del software libre, el conocimiento está al alcance de todos. Con este proyecto se pretende enseñar a realizar una tarea muy común en la robótica de servicio: \textit{seguir a un objetivo}. Además, conseguiremos fomentar la participación del alumnado o usuario a través de la plataforma académica Unibotics poniendo en práctica lo aprendido tanto en una simulación como en la realidad.




% -- SECCION OBJETIVOS
% ----------------------
\section{Objetivos}
\label{sec:objetivos}
El objetivo principal de este Trabajo Fin de Grado es desarrollar dos ejercicios educativos para seguir a una persona en \textit{Robotics Academy}. Estos ejercicios, una vez superen ciertos tests de calidad, se incorporarán a la plataforma \textit{Unibotics}. Un ejercicio consistirá en programar un robot Turtlebot2 simulado para seguir a una persona en un hospital, y el otro, en programar la misma tarea pero usando el robot real. Para hacer este proyecto realidad, se han marcado estos subobjetivos:

\begin{itemize}
	\item Crear soporte simulado del robot Turtlebot 2 en \textit{ROS Foxy}. La última versión simulada del Turtlebot2 se encuentra en ROS Noetic distribuido por \textit{The Construct}\footnote{\textbf{Turtlebot2 (Noetic):} \url{https://bitbucket.org/theconstructcore/turtlebot/src/noetic/}}. Profundizaremos más en el capítulo [\ref{cap:capitulo4}]
	\item Entender la infraestructura de \textit{Robotics Academy} y aprender a desarrollar un nuevo ejercicio, diseñando su plantilla web (frontend) y sus propios ficheros personales (exercise.py, interfaces con ROS, y módulos HAL y GUI). Veremos más en los capitulos [\ref{cap:capitulo5}] y [\ref{cap:capitulo6}]
	\item Diseñar un escenario en el simulador para el ejercicio \textit{Follow Person simulado} (incorporando un hospital de un repositorio externo de AWS) y programando un \textit{plugin} \footnote{\textbf{plugin}: Complemento que añade una funcionalidad extra o mejora a un programa} que permita controlar a una persona simulada desde el navegador web [\ref{sec:teleoperador}]
	\item Controlar el robot real a través de un contenedor \textit{Docker}. Un robot que se conecta a un portatil abre muchos dispositivos (/dev) que necesitan ser mapeados para ser utilizados por el contenedor [\ref{cap:capitulo6}]
	\item Diseñar una solución de referencia para cada ejercicio. De este modo demostramos la posibilidad de usar el nuevo ejercicio como recurso educativo para futuros cursos académicos en el grado de Robótica Software y en otros ámbitos. Además, enseñamos un posible modo de resolver el problema Sigue-Personas [\ref{sec:sigue_personas_simulado}] [\ref{sec:sigue_personas_real}]
\end{itemize}



% -- SECCION METODOLOGÍA
% ------------------------
\section{Metodología}
\label{sec:metodologia}
El TFG comenzó en octubre y finalizó en mayo. Durante estos meses hemos seguido el siguiente protocolo:

\begin{itemize}
	\item Reuniones semanales con el tutor de TFG para comentar los avances y recibir retroalimentación. De esta manera abordábamos varios problemas buscando otras soluciones.
	\item Se ha usado la plataforma \textit{Slack}\footnote{\textbf{Slack:} \url{https://slack.com/intl/es-es/}} para estar en comunicación con los miembros de Robotics Academy y Unibotics.
	\item Anotaciones semanales (con algunas excepciones dependiendo de la situación del curso académico) en un \textit{Blog}\footnote{\textbf{Weblog}: \url{https://roboticslaburjc.github.io/2021-tfg-carlos-caminero/}} de las pruebas realizadas o logros conseguidos. De esta manera mostramos al público el progreso del trabajo desde sus comienzos.
	\item Todas las pruebas realizadas así como los ficheros fuente del proyecto se han ido subiendo, a lo largo de estos meses, en un repositorio habilitado de \textit{Robotics Lab URJC}\footnote{\textbf{Repositorio TFG:} \url{https://github.com/RoboticsLabURJC/2021-tfg-carlos-caminero}}
\end{itemize}

El primer mes se correspondió a una \textit{etapa de entrenamiento} donde se empezó a probar distintas tecnologías para elegir correctamente el tema a desarrollar: Behavior Trees, Groot, aplicaciones en ROS 2, ROS Bridge, etc.\\

Una vez elegido el tema del trabajo, fue necesario aprender varias herramientas y tecnologías para su elaboración:
\begin{itemize}
	\item Aprender a construir imágenes Docker personalizados.
	\item Profundizar en XACRO y plugins a través de la página oficial de ROS Foxy \footnote{\textbf{Ros Foxy doc}: \url{https://docs.ros.org/en/foxy/}}.
	\item Para profundizar en HTML, CSS y Javascript se ha consultado el curso del profesor \textit{Juan González Gomez} (también conocido como \textit{Obijuan}) de la asignatura de Construcción de Servicios y Aplicaciones Audiovisuales en Internet (CSAAI) \footnote{\textbf{CSAAI}: \url{https://github.com/myTeachingURJC/2020-2021-CSAAI/wiki}}.
\end{itemize}

Una vez adquiridas las competencias necesarias, se desarrolló el entorno de Gazebo y se programó el plugin para mover a una persona simulada con el teclado. Después desarrollamos la plantilla web para la programación del ejercicio y la integramos en un contenedor Docker.\\

Al tener que usar un robot real, contaba con todo el hardware necesario tanto por parte de mi tutor (todo tipo de cámaras para hacer pruebas) como por parte del laboratorio de Robótica de la ETSIT (el robot Turtlebot2, y los RPLIDAR).


% -- SECCIÓN PLAN DE TRABAJO
% ----------------------------
\section{Plan de Trabajo}
\label{sec:plan_trabajo}
A lo largo de estos meses, la planificación del TFG ha sido la siguiente:

\begin{enumerate}
	\item \textbf{Etapa de Entrenamiento}. Conociendo las herramientas y posibilidades antes de decidir el tema del que consiste este proyecto: ROS, ROS2, ROS Bridge, Docker, BTs, etc.
	\item \textbf{Inicio} del TFG. Una vez conocido el tema, comenzar a organizar el proyecto, preparando el entorno ROS, programando el plugin y migrando el Turtlebot2 a ROS Foxy.
	\item Programación de la \textbf{infraestructura} de los ejercicios. Usando como referencia otros ejercicios de Robotics Academy como por ejemplo \textit{Color Filter} \footnote{\textbf{Color filter (foxy github):} \url{https://github.com/JdeRobot/RoboticsAcademy/tree/foxy/exercises/static/exercises/color_filter/web-template}} o \textit{Follow Line} \footnote{\textbf{Follow Line (noetic github):} \url{https://github.com/JdeRobot/RoboticsAcademy/tree/master/exercises/static/exercises/follow_line/web-template}}, y, adaptándolo a ROS 2.
	\item Programación de una \textbf{solución de referencia} para cada ejercicio Sigue Personas de Robotics Academy.
	\item \textbf{Memoria de Trabajo Fin de Grado}. La redacción de la presente memoria se realizó durante los últimos 4 meses del curso. 
\end{enumerate}


