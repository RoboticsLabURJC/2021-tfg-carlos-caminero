\chapter{Objetivos y Metodología de Trabajo}
\label{cap:capitulo2}

Una vez descritos en el primer capítulo la motivación y el marco en el que se encuadra este TFG (el contexto de la robotica, la Inteligencia Artificial y las plataformas educativas), en este segundo capítulo fijamos los objetivos concretos que nos hemos planteado para este proyecto.\\

% -- SECCION OBJETIVOS
% ----------------------
\section{Objetivos}
\label{sec:objetivos}

La importancia de los robots de servicio ha ido aumentando en los últimos años, sobre todo en campos como la salud y el entretenimiento. Con este proyecto se pretende enseñar a realizar una tarea muy común en este sector: \textit{seguir a un objetivo}.\\

El objetivo principal de este TFG es desarrollar dos ejercicios educativos en Robotics Academy para programar un robot que realice la tarea de seguir a una persona. Estos ejercicios, una vez superen ciertos tests de calidad, se incorporarán a la plataforma Unibotics. Un ejercicio consistirá en programar un robot TurtleBot2 simulado para realizar dicha tarea en un hospital, y el otro, usando el robot real. Con ello, conseguiremos fomentar la participación del alumnado o usuario con dos ejercicios donde tendrán que poner en práctica lo aprendido tanto en una simulación como en el mundo real. Para lograr esta meta, se han marcado estos subobjetivos:

\begin{enumerate}
	\item Crear soporte simulado del robot TurtleBot2 en ROS2 Foxy. La última versión simulada del TurtleBot2 se encuentra en ROS Noetic distribuido por The Construct\footnote{\textbf{Turtlebot2 (Noetic):} \url{https://bitbucket.org/theconstructcore/turtlebot/src/noetic/}}. Profundizaremos más en el capítulo \ref{cap:capitulo4}
	\item Entender la infraestructura de Robotics Academy y aprender a desarrollar un nuevo ejercicio, diseñando su plantilla web (frontend) y sus propios ficheros específicos (exercise.py, interfaces con ROS, y módulos HAL y GUI). Veremos más en el capítulo \ref{cap:capitulo5}
	\item Diseñar un escenario en el simulador para el ejercicio Sigue-Persona simulado (incorporando un hospital de un repositorio externo de AWS) y programando un \textit{plugin} \footnote{\textbf{plugin}: Complemento que añade una funcionalidad extra o mejora a un programa} que permita controlar a una persona simulada desde el navegador web.
	\item Controlar el robot real a través de un contenedor Docker. Un robot que se conecta a un portatil abre muchos dispositivos (/dev) que necesitan ser mapeados para ser utilizados desde dentro del contenedor.
	\item Diseñar una solución de referencia para cada ejercicio. De este modo demostramos la posibilidad de usar el nuevo ejercicio como recurso educativo para futuros cursos académicos en el grado en Ingeniería de Robótica Software y en otros ámbitos. Además, enseñamos un posible modo de resolver el problema Sigue-Persona en el capítulo \ref{cap:capitulo6}
\end{enumerate}



% -- SECCION METODOLOGÍA
% ------------------------
\section{Metodología}
\label{sec:metodologia}
El TFG comenzó en octubre de 2021 y finalizó en mayo de 2022. Durante estos meses se ha seguido el siguiente protocolo:

\begin{itemize}
	\item Reuniones semanales con el tutor de TFG para comentar los avances y recibir retroalimentación. De esta manera abordábamos varios problemas buscando otras soluciones.
	\item Se ha usado la plataforma Slack\footnote{\textbf{Slack:} \url{https://slack.com/intl/es-es/}} para estar en comunicación con el equipo de desarolladores de Robotics Academy y de Unibotics.
	\item Anotaciones semanales en un \textit{Blog}\footnote{\textbf{Weblog}: \url{https://roboticslaburjc.github.io/2021-tfg-carlos-caminero/}} de las pruebas realizadas o logros conseguidos. De esta manera mostramos el progreso del trabajo desde sus comienzos.
	\item Todas las pruebas realizadas así como el código fuente del proyecto se han ido subiendo, a lo largo de estos meses, en un repositorio en Github habilitado dentro de la cuenta de \textit{Robotics Lab URJC}\footnote{\textbf{Repositorio TFG:} \url{https://github.com/RoboticsLabURJC/2021-tfg-carlos-caminero}}
\end{itemize}


% -- SECCIÓN PLAN DE TRABAJO
% ----------------------------
\section{Plan de Trabajo}
\label{sec:plan_trabajo}
A lo largo de estos meses, la planificación del TFG ha sido la siguiente:

\begin{enumerate}
	\item Etapa de Entrenamiento. Se correspondió al primer mes donde se empezó a probar distintas tecnologías para elegir correctamente el tema a desarrollar: Behavior Trees, Groot, aplicaciones en ROS2, ROS Bridge, etc.
	\item Inicio del TFG. Una vez conocido el tema, comenzamos a organizar el proyecto, preparando el entorno ROS, programando un \textit{plugin} para mover una persona simulada y migrando el TurtleBot2 a ROS2 Foxy. También fue necesario aprender varias herramientas y tecnologías para su elaboración:
	\begin{itemize}
		\item Aprender a construir imágenes Docker personalizadas.
		\item Profundizar en Xacro y \textit{plugins} a través de la página oficial de ROS2 Foxy\footnote{\textbf{Ros Foxy doc}: \url{https://docs.ros.org/en/foxy/}}.
		\item Para profundizar en HTML, CSS y JavaScript se ha consultado el curso del profesor \textit{Juan González Gomez} de la asignatura de Construcción de Servicios y Aplicaciones Audiovisuales en Internet (CSAAI) \cite{CSAAI}.
	\end{itemize}
	\item Programación de la infraestructura de los ejercicios.
	\begin{itemize}
		\item Primero se empezó con el ejercicio Sigue-Persona simulado: se implementó el escenario del simulador Gazebo y se programó un \textit{plugin} para mover a una persona simulada con el teclado. Después desarrollamos la plantilla web para la programación del ejercicio y la integramos en un contenedor Docker, usando como referencia otros ejercicios de Robotics Academy como por ejemplo \textit{Color Filter} \footnote{\textbf{Color filter (foxy github):} \url{https://github.com/JdeRobot/RoboticsAcademy/tree/foxy/exercises/static/exercises/color_filter/web-template}} o \textit{Follow Line} \footnote{\textbf{Follow Line (noetic github):} \url{https://github.com/JdeRobot/RoboticsAcademy/tree/master/exercises/static/exercises/follow_line/web-template}}, y, adaptándolo a ROS2.
		\item De manera similar, hicimos lo mismo con el ejercicio Sigue-Persona Real. Al tener que usar un robot real, contaba con todo el hardware necesario tanto por parte de mi tutor (todo tipo de cámaras para hacer pruebas) como por parte del laboratorio de Robótica de la ETSIT (el robot TurtleBot2, y los láseres RPLIDAR).
	\end{itemize}
	\item Programación de una solución de referencia para cada ejercicio Sigue-Persona de Robotics Academy.
	\item Memoria de Trabajo Fin de Grado. La redacción de la presente memoria se realizó durante los últimos 4 meses del curso. 
\end{enumerate}


