\chapter{Conclusiones}
\label{cap:Conclusiones}

Concluímos este Trabajo Fin de Grado con las conclusiones y un resumen con los objetivos que hemos alcanzado, además de algunas posibles líneas para futuras derivaciones, investigaciones o ampliaciones que superen las metas alcanzadas en este proyecto

\section{¿Qué ha aportado este trabajo?}
\label{sec:aportaciones}

\begin{itemize}
	\item Migración de un modelo Turtlebot2 simulado y adaptado de ROS Noetic a ROS Foxy. Incorporación de los ficheros fuente en el repositorio oficial de \textbf{Custom Robots} \footnote{\url{https://github.com/JdeRobot/CustomRobots/tree/foxy-devel}} (rama foxy)
	\item La primera vez que se utiliza un \textbf{robot real} en un ejercicio que usa la infraestructura Robotics Academy / Unibotics.
	\item Cración de un \textbf{plugin} en ROS2 para controlar una persona simulada en Gazebo usando los botones del teclado a través del Frontend de la plantilla web
	\item Desarrollo de \textbf{2 nuevos ejercicios} para su implementación en la plataforma universitaria Unibotics. Actualmente han sido incorporados al set de ejercicios de Robotics Academy: \textbf{Follow Person}\footnote{\url{http://jderobot.github.io/RoboticsAcademy/exercises/MobileRobots/follow_person}} y \textbf{Real Follow Person}\footnote{\url{http://jderobot.github.io/RoboticsAcademy/exercises/MobileRobots/real_follow_person}}.
	\item Desarrollo de una solución robusta y funcional Sigue-Personas para los 2 nuevos ejercicios.
	\item Uso de un modelo de red neuronal óptimo para arquitecturas y software de bajo rendimiento computacional (ejemplo: Docker, Raspberry Pi).
\end{itemize}

\section{Competencias adquiridas}
\label{sec:competencias}

A continuación muestro los conocimientos y competencias adquiridas con la realización de este TFG:
\begin{itemize}
	\item Conocimiento avanzado de Docker: creación de imágenes y uso de contenedores
	\item Profundización en HTML, CSS y Javascript.
	\item Ampliación de conocimientos en ROS1, ROS2 y ROS Bridge.
	\item Creación de plugins para Gazebo.
	\item Amplicación de conocimientos en URDF y XACRO.
	\item Funcionamiento del Frontend y Backend de Robotics Academy.
	\item Creación de una solución más robusta en el problema robótico de seguir a una persona: \textbf{tracking, VFF}.
	\item Correcta metodología para trabajar en proyectos de software libre en Github.
\end{itemize}

\section{Líneas futuras}
\label{sec:lineas_futuras}

(TODO)

