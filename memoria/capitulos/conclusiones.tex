\chapter{Conclusiones}
\label{cap:Conclusiones}

Concluímos este Trabajo Fin de Grado con las conclusiones y un resumen con los objetivos que hemos alcanzado, además de algunas posibles líneas para futuras derivaciones, investigaciones o ampliaciones que superen las metas alcanzadas en este proyecto

\section{¿Qué ha aportado este trabajo?}
\label{sec:aportaciones}

\begin{itemize}
	\item Migración de un modelo TurtleBot2 simulado y adaptado de ROS Noetic a ROS2 Foxy. Incorporación de los ficheros fuente en el repositorio oficial de Custom Robots \footnote{\url{https://github.com/JdeRobot/CustomRobots/tree/foxy-devel}} (rama foxy)
	\item La primera vez que se utiliza un \textit{robot real} en un ejercicio que usa la infraestructura Robotics Academy / Unibotics.
	\item Cración de un \textit{plugin} en ROS2 para controlar una persona simulada en Gazebo usando los botones del teclado a través del frontend de la plantilla web
	\item Desarrollo de dos nuevos ejercicios para su implementación en la plataforma universitaria Unibotics. Actualmente han sido incorporados al set de ejercicios de Robotics Academy: Sigue-Persona Simulado\footnote{\url{http://jderobot.github.io/RoboticsAcademy/exercises/MobileRobots/follow_person}} y Sigue-Persona Real\footnote{\url{http://jderobot.github.io/RoboticsAcademy/exercises/MobileRobots/real_follow_person}}.
	\item Desarrollo de una \textit{solución} robusta y funcional Sigue-Personas para los 2 nuevos ejercicios.
	\item Incoporación de un modelo de red neuronal óptimo para arquitecturas y software de bajo rendimiento computacional (ejemplo: Docker, Raspberry Pi) para Robotics Academy.
\end{itemize}\

\section{Competencias adquiridas}
\label{sec:competencias}

Durante la realización del TFG se han ido adquiridendo varios conocimientos y competencias sobre distintas tecnologías:

\begin{itemize}
	\item Conocimiento avanzado de Docker: creación de imágenes y uso de contenedores
	\item Profundización en HTML, CSS y JavaScript.
	\item Ampliación de conocimientos en ROS1, ROS2 y ROS Bridge.
	\item Creación de \textit{plugins} para Gazebo.
	\item Amplicación de conocimientos en URDF y Xacro.
	\item Funcionamiento del frontend y backend de Robotics Academy.
	\item Creación de una solución más robusta en el problema robótico de seguir a una persona utilizando algoritmos como \textit{tracking} y \textit{VFF}.
	\item Correcta metodología para trabajar en proyectos de software libre en Github.
\end{itemize}\

\section{Líneas futuras}
\label{sec:lineas_futuras}

Una vez lograda la meta principal de este proyecto, han quedado abiertas varias ramas interesantes para mejorar o investigar:

\begin{itemize}
	\item Sería posible incorporar varios modelos de redes neuronales ligeros parecidos a SSD Inception V2 para ejercicios de \textit{Deep Learning}.
	\item Crear nuevos ejercicios para Robotics Academy usando el TurtleBot2 real o su modelo simulado.
	\item Encontrar soporte para cámaras RGB-D en ROS2 Foxy que puedan ser utilizadas en \textit{Robotics Academy} para aprovechar la característica de estimar la distancia de los objetos presentes en los fotogramas.
	\item En relación al apartado anterior, sería interesante poder mejorar la solución Sigue Personas aprovechando la profundidad de una cámara RGB-D. De esta manera, al identificar al objetivo podríamos lanzar una transformada desde el \textit{frame base\_footprint} y seguir constantemente a la persona gracias a su frame dinámico. Con ello, mejoraríamos en precisión y sería más difícil perder a la persona. También se podría usar la profundidad para determinar el módulo del vector de atracción al utilizar VFF.
\end{itemize}

