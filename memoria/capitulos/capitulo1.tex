\chapter{Introducción}
\label{cap:capitulo1}
\setcounter{page}{1}

La tarea Sigue-Personas es una de las más empleadas y solicitadas en los Robots de Servicio para multitud de sectores que incorporan soluciones robóticas. El siguiente TFG propone desarrollar dos nuevos ejercicios para la plataforma Unibotics con el objetivo de que los usuarios programen un robot Turtlebot2 tanto en simulado como en real para que realice dicha tarea.\\

Este primer capítulo trata de presentar el estado actual de varios campos de la Robótica que están directamente relacionados con este proyecto para poner en contexto al lector y facilitar su lectura.\\

Por un lado presentaremos la Robótica Móvil, un sector de la robótica que está en constante crecimiento y muy presente en la Robótica de Servicio; por otra parte, introduciremos las Redes Neuronales, un avance significativo en IA que ha permitido realizar aplicaciones muy robustas basadas en la percepción y el razonamiento; y por último hablaremos de la Robótica Educativa, e introduciremos varias plataformas como TheConstruct, Robotics Academy, o Unibotics siendo estas dos últimas las que incorporarán el ejercicio educativo Sigue-Persona.

\section{Robótica Móvil}
\label{sec:robotica_movil}

La robótica es la ciencia que engloba varias ramas tecnológicas o disciplinas, con el objetivo de diseñar máquinas (``robots'') que sean capaces de realizar tareas automatizadas o de simular el comportamiento humano o animal, en función de la capacidad de su software. \cite{revistaderobots}. Su término se remonta a la obra de ciencia ficción escrita por Isaac Asimov: Yo, Robot. \\

Su evolución tuvo lugar en la segunda mitad del siglo XX a partir de varios avances:\\
- En 1952 aparece la primera máquina de control numérico del MIT que era capaz de automatizar algunas tareas industriales.\\
- En 1961, la compañía Unimates introdujo el primer robot industral en la General Motors.\\
- En 1966, comenzó el desarrollo del primer robot móvil llamado Shakey [Nilsson, 1984].

\begin{figure} [h!]
  \begin{center}
    \includegraphics[width=5cm]{imagenes/shakey.jpg}
  \end{center}
  \caption{Shakey (1966-1972) \cite{shakey-the-robot}}
  \label{fig:shakey}
\end{figure}\

Un robot móvil es aquel robot capaz de desplazarse mediante ruedas o patas sobre su entorno.

\section{Redes Neuronales}
\label{sec:redes_neuronales}

Según \emph{Arthur Samuel}, el Aprendizaje Automático o Machine Learning es el ``campo de estudio que otorga a los ordenadores la capacidad de aprender sin ser programados explícitamente''. Para ello se emplean diversas técnicas dependiendo del objetivo deseado: Regresión Lineal, Regresión Logística, K-NN, K-Means, SVM, Redes Neuronales, Q-Learning, etc.
\\

Cuando incorporamos una Red Neuronal, lo que queremos es que nuestro programa o aplicación sepa \textbf{clasificar} unos datos de entrada con su correspondiente salida, habiendo entrenado previamente un modelo computacional inspirado en el cerebro humano. Para ello se le proporciona un set de datos de entrenamiento, donde indicamos, por cada muestra, a qué clase pertenece (Aprendizaje Supervisado). De esta forma, ajustando los valores de unos pesos que se utilizan en el entrenamiento, conseguiremos un nivel de clasificación determinado.\\

Una aplicación de Redes Neuronales es la detección mediante Vision Artificial de objetos. Para ello, se usan modelos de redes de varias capas con un gran número de neuronas (Deep Learning) y de esta forma conseguimos dada una imagen detectar ciertos objetos solicitados como puede ser fruta, coches, utensilios o incluso hasta personas y animales.\\

Típicamente, una Red Neuronal consta de un conjunto de capas formado por varias neuronas que se encuentran interconectadas. Suelen dividirse en 1 capa de entrada, 1 o más capas ocultas y 1 capa de salida. Pero, ¿cómo funciona una neurona?\\


\begin{figure} [h!]
  \begin{center}
    \includegraphics[width=15cm]{imagenes/neurona.png}
  \end{center}
  \caption{Una Neurona \cite{AIMA}}
  \label{fig:neurona}
\end{figure}\

Una neurona está formada pos los siguientes elementos:
\begin{itemize}
	\item \textbf{Valores de Entrada}: Puede ser tanto los valores de las \textbf{características} de cada muestra como la salida de la neurona de una capa anterior.
	\item \textbf{Pesos}: Son unos valores que se optienen al final de la etapa de entrenamiento.
	\item \textbf{Sumatorio}: Esta formado por los productos de los Valores de Entrada por sus correspondientes Pesos
	\item \textbf{Función de Decisión/Activación}: Es una función matemática que se aplica al sumatorio.
	\item \textbf{Salida}: Es el valor de la función de decisión. Dependiendo de la función aplicada, la salida será distinta.
\end{itemize}

\section{Educación en Robótica}
\label{sec:educacion_robotica}

\subsection{TheConstruct}
\label{sec:the_construct}

\subsection{Robotics Academy}
\label{sec:robotics_academy}

\subsection{Unibotics}
\label{sec:unibotics}


