\chapter{Introducción}
\label{cap:capitulo1}
\setcounter{page}{1}

La Robótica de Servicio es la rama de la robótica que presta ayuda al ser humano. Surge a principios de los años 80, como necesidad de extender la robótica más alla de los entornos industriales, para abordar otras tareas que aporten servicio a la sociedad.\\

Dependiendo del servicio que proporciona el robot o aplicación robótica, podemos hablar de robótica de entretenimiento, de salud, de campo, de logística, o de limpieza.\\

La Robótica de Entretenimiento es aquella aplicada a la enseñanza y al ocio principalmente. El presente Trabajo Fin de Grado trata del desarrollo de 2 ejercicios para una plataforma educativa de robótica, por tanto, debido a su fin pedagógico es interesante mostrar cuál es actualmente el estado del arte de la Robótica de Entretenimiento.

\section{Estado del Arte}
\label{sec:estado_arte}

(TODO estado del arte)
(TODO finalizo el estado del arte y doy paso a Robotics Academy)

\section{Robotics Academy}
\label{sec:robotics_academy}

(TODO Robotics Academy. Arquitectura, ejercicios...)

\section{Unibotics}
\label{sec:unibotics}

(TODO Unibotics. Arquitectura, ejercicios...)
