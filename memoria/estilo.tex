\documentclass[oneside,a4paper,12pt]{book}

\usepackage[spanish]{babel}
\usepackage[utf8]{inputenc}
\usepackage{geometry}
\usepackage{makeidx}
\usepackage{url}
\usepackage{graphicx}
\usepackage{color}
\usepackage{caption}
\usepackage{acronym}
\usepackage{hyphenat}
\usepackage{a4wide}
\usepackage[normalsize]{subfigure}
\usepackage{float}
\usepackage{titlesec}
\usepackage[Lenny]{fncychap}
\usepackage{listings} % para poder hacer uso de "listings" propios (p.ej. códigos)
\usepackage{eurosym} % para poder usar el símbolo del euro con \euro {xx}
\usepackage{hyperref} % TODO: añade la opción hidelinks para imprimirlo (los enlaces no aparecerán resaltados)

% Para que no parta las palabras
\pretolerance=10000

\newcommand{\bigrule}{\titlerule[0.5mm]} \titleformat{\chapter}[display] % cambiamos el formato de los capítulos
{\bfseries\Huge} % por defecto se usaron caracteres de tamaño huge en negrita
{% contenido de la etiqueta 
\titlerule % línea horizontal 
\filright % texto alineado a la derecha 
\Large\chaptertitlename\ % capítulo e índice en tamaño large
\Large % en lugar de 
\Huge \Large\thechapter} 
{0mm} % espacio mínimo entre etiqueta y cuerpo
{\filright} % texto del cuerpo alineado a la derecha
[\vspace{0.5mm} \bigrule] % después del cuerpo, dejar espacio vertical y trazar línea horizontal gruesa
\geometry{a4paper, left=3.5cm, right=2cm, top=3cm, bottom=2cm, headsep=1.5cm}
