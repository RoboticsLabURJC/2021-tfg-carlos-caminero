\chapter*{Resumen}

La Robótica es un sector en constante crecimiento. Cada vez es más importante el perfil del Ingeniero en Robótica Software para la programación de Robots y Automatismos en esta nueva era digital. Por ello, es necesario la correcta formación del Ingeniero a través de la accesibilidad y disponibilidad de los recursos educativos que fomentarán su aprendizaje autodidacta.\\

El objetivo del siguiente Trabajo Fin de Grado (TFG) es desarrollar dos nuevos ejercicios educativos para la plataforma de enseñanza universitaria Robotics Academy consistentes en programar un robot modelo TurtleBot2 de Yujin para que sea capaz de seguir a una persona tanto en un entorno real como en un entorno simulado (un hospital). Con ello, los alumnos y otros usuarios adquirirán las destrezas necesarias para la programación de una tarea muy solicitada en la Robótica de Servicio. Aprenderán a seguir a una persona usando Redes Neuronales que ayudarán a la detección visual de personas en imágenes\\

La incorporación de estos dos nuevos ejercicios se aplica a una nueva rama de desarrollo para la plataforma Robotics Academy que usa la distribución ROS2 Foxy. Además, se ha logrado la migración y adaptación del modelo simulado del Turtlebot2 de ROS Noetic a ROS Foxy para futuros usos. Se ha preparado la página web operativa correspondiente a cada ejercicio y se han desarrollado las respectivas soluciones de referencia que combinan percepción visual robusta y navegación con el algoritmo VFF.