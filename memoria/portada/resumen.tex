\chapter*{Resumen}

La Robótica es un sector en constante crecimiento.
Cada vez es más importante el perfil del Ingeniero en Robótica Software para la programación de Robots y Automatismos en esta nueva era digital.
Por ello, es necesario la correcta formación del Ingeniero a través de la accesibilidad
y disponibilidad de los recursos educativos que fomentarán su aprendizaje autodidacta.\\

El objetivo del siguiente Trabajo Fin de Grado (TFG) es desarrollar e implementar 2 nuevos ejercicios educativos para la plataforma
de enseñanza universitaria Unibotics consistentes en programar un robot modelo Turtlebot 2 de Yujin para que sea capaz de seguir a
una persona tanto en un entorno real como en un entorno simulado (un hospital). Con ello, los alumnos y otros usuarios adquirirán las destrezas necesarias 
para la programación de una tarea muy solicitada en la Robótica de Servicio. Aprenderán a seguir a una persona usando un framework de
Redes Neuronales que ayudarán a la detección.\\

La incorporación de estos 2 nuevos ejercicios se aplicarán a una nueva rama de desarrollo para la plataforma Unibotics que usa la distribución ROS Foxy como base.
Además, se habrá logrado la migración y adaptación del modelo simulado del Turtlebot2 de ROS Noetic a ROS Foxy para futuros usos.